\pagestyle{plain}
\chapter{Explanation/Main Part}

\noindent Math font:
\[
 \mathcal{B} = \{B_{\alpha} \in \mathcal{T}\, |\,  U = \bigcup B_{\alpha} \forall U \in \mathcal{T} \}
\]
\begin{align*}
    3x^2 \in R \subset Q \\
    \mathnormal{3x^2 \in R \subset Q} \\
    \mathrm{3x^2 \in R \subset Q} \\
    \mathit{3x^2 \in R \subset Q} \\
    \mathbf{3x^2 \in R \subset Q} \\
    \mathsf{3x^2 \in R \subset Q} \\
    \mathtt{3x^2 \in R \subset Q}
\end{align*}

\noindent Inline math code: \(x^2 + y^2 = z^2\) \\
Also use \texttt{\$...\$}: $E=mc^2$ \\
Or: \verb|\begin{math}...\end{math}|: \begin{math}\frac{1}{2}\end{math} \\

\noindent Display math code: \[ x^n + y^n = z^n \] 
\begin{displaymath}
    \sqrt{x^2+1}
\end{displaymath}
\begin{equation*}
    \sqrt{x^2+1}
\end{equation*}

\noindent Refer to equations with \texttt{equation} or \texttt{multiline} without \texttt{*}:
\begin{equation} \label{example}
    \sqrt{x^2+1}
\end{equation}
\noindent Compare with \ref{example}.\\

\noindent For longer equations use \texttt{multiline}:
\begin{multline*}
    p(x) = 3x^6 + 14x^5y + 590x^4y^2 + 19x^3y^3\\ 
    - 12x^2y^4 - 12xy^5 + 2y^6 - a^3b^3
\end{multline*}

\newpage

\noindent Sum inside text: $\sum_{n=1}^{\infty} 2^{-n} = 1$ \\
Sum outside text: \[ \sum_{n=1}^{\infty} 2^{-n} = 1 \]

\noindent Product inside text: $\prod_{i=a}^{b} f(i)$ \\
Product outside text: \[ \prod_{i=a}^{b} f(i) \]


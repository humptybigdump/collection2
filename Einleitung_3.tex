\pagestyle{plain}
\chapter{Einleitung}

\noindent Diese Vorlage dient zur Anfertigung der wissenschaftlichen Hausarbeit im Rahmen der Lehrveranstaltung Arbeitstechniken im Maschinenbau (ATM). 
Die Gestaltungsmerkmale orientieren sich an den Richtlinien zur Manuskriptgestaltung der Deutschen Gesellschaft für Psychologie (DGPs) von 2007 und an den Richtlinien der American Psychological Association (APA). 
Beide Richtlinien können in der ATM-Lernumgebung auf ILIAS abgerufen werden. 
Darin finden Sie auch umfangreiche Erläuterungen zur Gestaltung von Tabellen, Abbildungen und Zitationen sowie alle Formatierungsgrundlagen und Typographien. 
Das oberste Leitprinzip innerhalb einer wissenschaftlichen Arbeit ist Konsistenz. 
Eine Arbeit sollte nicht nur inhaltlich widerspruchsfrei, sondern auch in formaler Hinsicht einheitlich gestaltet sein. 
Hat man sich für eine bestimmte Gestaltungsregel entschieden, so ist diese Regel in der ganzen Arbeit beizubehalten. 
In dieser Arbeit besitzen die Textseiten einen Seitenrand von 3~cm links, 2~cm rechts, 2~cm unten und 2,5~cm oben. 


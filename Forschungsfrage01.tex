\section{Forschungsfrage01 zum Thema (Vorname01 Nachname01)}

\subsection{Problemstellung}

Eine Tabelle wird wie folgt eingefügt (Tab. \ref{tab:Beispieltabelle}):

\begin{table}[H]
	\caption{Tabellentitel bzw. Tabellenbeschriftung.}	
	\begin{tabularx}{1\textwidth}{|l|p{0.33\textwidth}|X|}\hline
		Überschrift 1 & Überschrift 2 & Überschrift 3\\
		\hline\hline
		XXX & YYY & ZZZ\\
		\hline
		\multirow{4}{*}{XXX}
			& \multirow{3}{*}{YYY} & ZZZ \newline ZZZ \newline ZZZ\\ 
			\cline{2-3}
			& YYY & ZZZ\\
		\hline
	\end{tabularx}
	\label{tab:Beispieltabelle}
\end{table}

Die Tabellenbeschriftung steht immer oberhalb der Tabelle und nach der Tabelle wird eine Leerzeile angelegt, genau wie vor einer Abbildung (Abb. \ref{fig:Beispielabbildung}). Der Abbildungstitel steht immer unterhalb der Abbildung.

\begin{figure}[H]
	\centering
	\includegraphics[width=\linewidth]{images/Beispielabbildung}
	\caption{Abbildungstitel bzw. Abbildungsbeschriftung.}
	\label{fig:Beispielabbildung}
\end{figure}

\subsection{Zielsetzung}

\subsection{Erläuterung}

\subsection{Diskussion}

\subsection{Schlussfolgerung}
\pagestyle{plain}
\chapter{Erläterung/Inhalt}

\noindent Mengenschreibweise:
\[
 \mathcal{B} = \{B_{\alpha} \in \mathcal{T}\, |\,  U = \bigcup B_{\alpha} \forall U \in \mathcal{T} \}
\]
\begin{align*}
    3x^2 \in R \subset Q \\
    \mathnormal{3x^2 \in R \subset Q} \\
    \mathrm{3x^2 \in R \subset Q} \\
    \mathit{3x^2 \in R \subset Q} \\
    \mathbf{3x^2 \in R \subset Q} \\
    \mathsf{3x^2 \in R \subset Q} \\
    \mathtt{3x^2 \in R \subset Q}
\end{align*}

\noindent Mathematische Formeln im Text: \(x^2 + y^2 = z^2\) \\
Man kann auch \texttt{\$...\$} benutzen: $E=mc^2$ \\
Oder auch: \verb|\begin{math}...\end{math}| \begin{math}\frac{1}{2}\end{math} \\

\noindent Mathematische Formeln getrennt vom Text: \[ x^n + y^n = z^n \] 
\begin{displaymath}
    \sqrt{x^2+1}
\end{displaymath}
\begin{equation*}
    \sqrt{x^2+1}
\end{equation*}

\noindent Mit \texttt{equation} oder \texttt{multiline} ohne \texttt{*} kann man auf Formeln verweisen:
\begin{equation} \label{example}
    \sqrt{x^2+1}
\end{equation}
\noindent Siehe \ref{example} für die Formel.\\

\noindent Für längere Formeln \texttt{multiline} verwenden
\begin{multline*}
    p(x) = 3x^6 + 14x^5y + 590x^4y^2 + 19x^3y^3\\ 
    - 12x^2y^4 - 12xy^5 + 2y^6 - a^3b^3
\end{multline*}

\newpage

\noindent Summen im Text: $\sum_{n=1}^{\infty} 2^{-n} = 1$ \\
Summen ohne Text: \[ \sum_{n=1}^{\infty} 2^{-n} = 1 \]

\noindent Produkte im Text: $\prod_{i=a}^{b} f(i)$ \\
Produkte ohne Text: \[ \prod_{i=a}^{b} f(i) \]

